%%%%%%%%%%%%%%%%%%%%%%%%%%%%%%%%%%%%%%%%%%%%%%%%%%%%%%%%%%%%%%%%%%%%%%%%%%%%%%%%%%%%%%%%%%%%%%%%%%%%%%%%%%%%%%%%%%%%%%%%%%%%%%%%%%%%%%%%%%%%%%%%
%              This is a general scheme for my grad homework write-ups; the first draft was prepared during Jan 31 2015 %
%%%%%%%%%%%%%%%%%%%%%%%%%%%%%%%%%%%%%%%%%%%%%%%%%%%%%%%%%%%%%%%%%%%%%%%%%%%%%%%%%%%%%%%%%%%%%%%%%%%%%%%%%%%%%%%%%%%%%%%%%%%%%%%%%%%%%%%%%%%%%%%%

%\documentclass[10pt, letterpaper, twoside]{amsart}  %"twoside" allows left/even and right/odd page # and headers


%MATH PACKAGES------------------------------------------------------------------------------------------------------

% \usepackage{amsmath}       %this package invokes extra math commands, e. g., "pmatrix" and "text"
 %\usepackage{amsthm}        %this package invokes 4 possible thm styles, e. g., "dfn," "rem," etc.
 %\usepackage{amssymb}       %this package invokes ams fonts, e. g., "blackboard" and "fraktur"
 %\usepackage{latexsym}      %this package invokes new commands,e.g., the "\'" smaller and higher prime
 %\usepackage{enumerate}

 \theoremstyle{definition}  %or you can use "theorem" for italicized statement
   \newtheorem*{exercise1}{Exercise 08.26}
%----------------------------------------------------------------------------------------------------------------------

%\begin{document}

%------------------------------------------------------------------------------------------------------------------------
\begin{flushleft}               %this is for more traditional hw write-ups
  
  \end{flushleft}

%------------------------------------------------------------------------------------------------------------------------
\vspace{1cm}
%------------------------------------------------------------------------------------------------------------------------

%problem #1
\begin{exercise1}
  Let $X_1$,$X_2$,...,$X_n$ be a random sample from a distribution whose first four moments exist, and let 
  $$
    S_n^{\,2}=\sum_{i=1}^n(X_i-\bar{X})^2/(n-1).
  $$ 
  Show that $ S_n^{\,2}\to \sigma^2$ as $n\to \infty$.
\end{exercise1}

\begin{proof}[Proof]
   Consider property $ S_n^{\,2}=(\sum_{i=1}^n X_i{\,2}-n\bar{X}^2)/(n-1).$ Based on this equation, we have,
    \begin{align*}
     E(S_n^{\,2})&=E[\sum_{i=1}^n X_i^{\,2}-n\bar{X}^2]/(n-1)\\
                &=\frac{1}{n-1}[\sum_{i=1}^nE(X_i^{\,2})-nE(\bar{x}^2)]\\
                &=\frac{1}{n-1}[n(\mu^2+\sigma^2)-n(\mu^2+\frac{\sigma^2}{n})]\\
                &=\frac{1}{n-1}[(n-1)\sigma^2]\\
                &=\sigma^2
     \end{align*}
   And according to the Theorem $8.2.2$, we also have,
   $$ 
   Var(S_n^{\,2})=(\mu_4-\frac{n-3}{n-1}\sigma^4)/n, n>1
   $$
   With the Chebychev inequality, we get
    $$ 
    P[|S_n^{\,2}-\sigma^2|<\varepsilon]\geq\frac{1-(\mu_4-\frac{n-3}{n-1}\sigma^4)}{\varepsilon^2n}
    $$
    so that $\lim_{x \to \infty}P[|S_n^{\,2}-\sigma^2|<\varepsilon]=1.$\\
Therefore, according to the definition of Convergence in Probability, we have $S_n^{\,2}\overset{p}{\to}\sigma^2.$
    
\end{proof}

%------------------------------------------------------------------------------------------------------------------------

------------------------------------------------------------------------------------------------------------------


%\end{document}

%------------------------------------------------------------------------------------------------------------------------ 